\documentclass[varwidth]{standalone}

\begin{document}
    We have a pile of $2016$ cards and a hat. We take out one card, put it in the hat and then divide the remaining cards into two arbitrary non-empty piles. In the next step, we choose one of the two piles, we move one card from this pile to the hat and then divide this pile into two arbitrary non-empty piles.

    This procedure is repeated several times: in the $k$-th step ($k > 1$) we move one card from one of the piles existing after the step ($k - 1)$ to the hat and then divide this pile into two non-empty piles.

    Is it possible that after some number of steps we get all piles containing three cards each?
\end{document}