\documentclass[varwidth]{standalone}

\begin{document}
    On consid\`ere un quadrillage de taille $n \times n$ form\'e par $n^2$ carr\'es de c\^ot\'e $1$. On d\'efinit le centre d'un carr\'e comme \'etant le point d'intersection de ses diagonales.

    D\'eterminer le plus petit entier $m$ tel que, parmi n'importe quel nombre $m$ de carr\'es dans le quadrillage, on en a toujours quatre dont les centres sont les sommets d'un parall\'elogramme.
\end{document}