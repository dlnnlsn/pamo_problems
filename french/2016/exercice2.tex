\documentclass[varwidth]{standalone}

\begin{document}
    On dispose d'une pile de $2016$ cartes et d'un chapeau. On tire une carte, on la met dans le chapeau et on partage arbitrairement les cartes restantes en deux piles. A l'\'etape suivante, on choisit l'une des deux piles, on tire une carte de cette pile que l'on met dans le chapeau puis on partage \`a nouveau arbitrairement le restant en deux nouvelles piles.

    On r\'ep\`ete ce proc\'ed\'e plusieurs fois : \`a la $k$-i\`eme \'etape on tire une carte \`a partir de l'une des piles constitu\'ees \`a l'\'etape $(k - 1)$ et on la met dans le chapeau puis on partage \`a nouveau cette pile en deux.

    Est-il possible, apr\`es un certain nombre de r\'ep\'etitions de ce proc\'ed\'e, d'obtenir un nombre de piles de sorte que chacune d'elles contienne trois cartes ?
\end{document}