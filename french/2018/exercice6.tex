\documentclass[varwidth]{standalone}

\begin{document}
    Un cercle est divis\'e en $n$ secteurs ($n \geq 3$). Chaque secteur peut \^etre rempli soit par $1$ ou $0$. On choisit n'importe quel secteur $\mathcal{C}$ contenant $0$, on le change en $1$ et on change simultan\'ement les symboles $x$, $y$ dans les deux secteurs adjacents \`a $\mathcal{C}$ en leurs compl\'ementaires $1 - x$, $1 - y$. On r\'ep\`ete ce proc\'ed\'e tant qu'il existe un z\'ero dans un certain secteur. Dans la configuration initiale il existe un $0$ dans un seul secteur et des $1$ dans les autres secteurs. Pour quelles valeurs de $n$ peut-on finir ce proc\'ed\'e ?
\end{document}