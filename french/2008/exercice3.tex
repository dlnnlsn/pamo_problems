\documentclass[varwidth]{standalone}

\usepackage{enumerate}

\begin{document}
    Soient trois entiers positifs non nuls $a$, $b$ et $c$ tels que $a < b < c$. On consid\`ere les ensembles $A$, $B$, $C$ et $X$ d\'efinis de la fa\c con suivante:
    $A = \{1, 2, \dots, a\}$, $B = \{a + 1, a + 2, \dots, b\}$, $C = \{b + 1, b + 2, \dots, c\}$ et $X = A \cup B \cup C = \{1, 2, \dots, c\}$.

    D\'eterminer en fonction de $a$, $b$ et $c$ le nombre de fa\c cons de placer les \'el\'ements de $X$ dans trois bo\^ites de la mani\`ere suivante:
    
    \begin{center}
        \begin{tabular}{|c|c|c|}
            \hline
            Bo\^ite $1$ & Bo\^ite $2$ & Bo\^ite $3$ \\
            \hline
            $x$ & $y$ & $z$ \\
            \hline
        \end{tabular}
    \end{center}

    Sachant que:
    \begin{enumerate}[a)]
        \item $x \leq y \leq z$;
        \item les \'el\'ements de $B$ ne doivent pas \^etre mis dans la bo\^ite $1$
        \item les \'el\'ements de $C$ ne peuvent \^etre mis que dans la bo\^ite $3$.
    \end{enumerate}
\end{document}